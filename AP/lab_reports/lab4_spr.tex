\documentclass[11pt]{report}
\usepackage[T1]{fontenc}
\usepackage[polish]{babel}
\usepackage{mathptmx}
\usepackage[utf8]{inputenc}
\usepackage{lmodern}
\usepackage{enumitem}
\usepackage{setspace}
\usepackage{advdate}
\usepackage{fancyvrb}
\usepackage[none]{hyphenat}
\usepackage{amsmath}
\selectlanguage{polish}
\usepackage[left=2.5cm,top=2.5cm,right=2.5cm,bottom=2.5cm,bindingoffset=0.5cm]{geometry}

\title{Sprawozdanie AK2 lab3}
\date{\today}
\author{Jakub Sokołowski}
\newcommand{\LabNum}{4}
\setcounter{chapter}{\LabNum}
\begin{document}
\noindent
Jakub Sokołowski, 226080
  \hfill Wrocław, dn.\ {\AdvanceDate[-1]\today}\\
WT-N-07\hfill  prowadzący: Aleksandra Postawka\\
\vspace{1cm}
\begin{center}
  \begin{Large}
  	Laboratorium Architektury Komputerów\\
    \emph{(\LabNum) Funkcje i stos\\}
  \end{Large}
\end{center}


\section{Treść ćwiczenia}
\label{sec:tresc-cwiczenia}
\textbf{Zakres i program ćwiczenia:}\\
Tworzenie zwykłych i rekurencyjnych funkcji w assemblerze z wykorzystaniem stosu. Stos, to struktura danych typu \textit{LIFO - last in, first out}, w której dane są dokładane na wierzch stosu (instrukcja \textit{push}), i stamtąd też są pobierane (instrukcja \textit{pop}). Stos rośnie w odwrotnym kierunku do sterty - od wyższych adresów pamięci, do niższych. Dostęp do elementów stosu jest możliwy za pomocą rejestru \textit{\%rsp}, który wskazuje na wierzchołek stosu.\\\\
\textbf{Zrealizowanie zadania:}
\begin{itemize}[leftmargin=*,topsep=0pt]
	\item{Funkcja zwracająca indeks początku najdłuższego ciągu zer w danym łańcuchu znaków}
	\item{Funkcja rekurencyjna:
	\[
    f(x)=\left\{
                \begin{array}{ll}
                  x_0 = -2\\
                  x_n = 5x_{n-1} + 7
                \end{array}
              \right.
  \]
  	\begin{enumerate}[label=(\alph*)]
  		\item{Argument i wyniki przekazywane przez stos}
  		\item{Argument i wyniki przekazywane przez rejestry}
  	\end{enumerate}
  	}
\end{itemize}
\newpage
\section{Przebieg ćwiczenia}
\subsection{Funkcje w assemblerze}
W assemblerze funkcje deklarujemy w następujący sposób:
\begin{verbatim}
.type zero_str, @function
zero_str:
\end{verbatim}
Pierwsza linia mówi linkerowi, że symbol \textit{zero\_str} ma być traktowany jako funkcja. Jeśli cały kod programu znajduje się w jednym pliku program zadziałałby też bez tej linii.\\
Funkcje w assemblerze wykorzystują dwie specjalne instrukcje \textit{call} i \textit{ret}. Instrukcja \textit{call} ma dwa efekty: odkłada adres następnej instrukcji (adres powrotu) na stos, a następnie modyfikuje wskaźnik instrukcji \textit{\%rip}, tak aby wskazywał na początek wywoływanej funkcji. Instrukcja \textit{ret} ściąga wartość z wierzchołka stosu, i ustawia \textit{\%rip} na tę wartość.
\\Pierwszym krokiem, jaki musi wykonać funkcja, jest zapisanie obecnego rejestru bazowego \textit{\%rbp} na stosie. Rejestr ten służy do dostępu do argumentów funkcji i zmiennych lokalnych. Jest on także stałym odniesieniem do ramki stosu (parametrów, zmiennych i adresu powrotu funkcji). Nie można do tego wykorzystać \textit{\%rsp}, ponieważ może być on modyfikowany przez inne funkcje podczas wykonywania programu.
\\Przed powrotem z funkcji, należy zresetować stos, do stanu w jakim się znajdował zaraz po wywołaniu funkcji - pozbyć się obecnej ramki stosu, a przwrócić tą należącą do kodu wywołującego funkcję. 
\begin{verbatim}
movq %rbp, %rsp
popq %rbp
ret 
\end{verbatim}
Pierwsza instrukcja resetuje wskaźnik stosu, a druga wskaźnik bazowy. Teraz na górze stosu znajduje się adres powrotu, do którego powraca instrukcja ret.
\label{sec:przebieg-cwiczenia}
\subsection{Najdłuższy ciąg zer}
Pierwszym zadaniem było napisanie funkcji znajdującej indeks początkowy najdłuższego ciągu zer w danym łańcuchu znaków. Jako parametr funkcja przejmuje wskaźnik na łańcuch znaku. Parametr ten jest przekazywany przez stos, więc wywołanie tej funkcji wygląda następująco:
\begin{verbatim}
mov $IN_BUF, %rbx
  pushq %rbx
  call  zero_str
\end{verbatim}
Działanie samej funkcji jest bardzo proste - funkcja iteruje po buforze wskazanym, aż natrafi na zero, wtedy iteruje dalej, zwiększając licznik zer, aż natrafi na znak który nie jest zerem. Po zakończeniu ciągu, jego długość jest porównywana z najdłuższą dotychczasową, a jeśli jest większa, to długość i indeks początkowy ciągu (indeks obecny - długość) są zapamiętywane.
\begin{verbatim}
.type zero_str, @function
zero_str:
    pushq %rbp              # Umieszczenie %rbp na stosie
    movq  %rsp, %rbp        # Wskaźnik stosu wskazuje teraz na %rbp
    movq  16(%rbp), %rbx    # Pierwszy argument funkcji do %rbx

    # Iteracja po buforze
    mov $0, %rdi            # Obecny indeks w buforze
    mov $0, %rax            # Obecny znak
    mov $0, %rcx            # Licznik zer
    mov $-1, %r10           # Jeśli nie ma ciągu zer, wynik to -1
    find_zero_loop:
        mov (%rbx, %rdi, 1), %al
        inc %rdi
        cmp $'0', %al       
        je count_zeros      # Znak jest zerem
        cmp $NEWLINE, %al    
        je end_zero_str     # Znak '\n' - koniec bufora
        jmp find_zero_loop  # Inny znak, następna iteracja
        count_zeros:
            inc %rcx        # Zwiększenie licznika zer
            count_loop:           
                mov (%rbx, %rdi, 1), %al
                inc %rdi
                cmp $'0', %al
                je count_zeros
                # Koniec ciągu zer, sprawdzenie długości 
                cmp %r8, %rcx
                jl find_zero_loop
            new_record:
                # Długość jest większa niż obecna największa, zapisanie
                # obecnego indeksu
                mov %rcx, %r8
                mov $0, %rcx
                dec %rdi
                mov %rdi, %r10
                sub %r8, %r10
                jmp find_zero_loop
    end_zero_str:
      movq %r10, %rax
      movq %rbp, %rsp
      popq %rbp
      ret 
\end{verbatim}

\subsection{Zapis w kolejności little endian}
\subsection{}
Po zapisaniu do pamięci liczb w poprawnej kolejności, można przystąpić do dodawania. Żeby zapewnić poprawną propagację przeniesienia, wykorzystywana jest instrukcja adc, która bierze pod uwagę flage przeniesienia (CF). Ta sama flaga jest modyfikowana podczas instrukcji \textbf{cmp} która jest wykonywana podczas pętli. W celu zachowania wartości CF, po dodaniu liczb rejestr flagowy umieszczany jest na stosie za pomocą instrukcji \textbf{pushfq}, a przed dodaniem jest pobierany ze stosu za pomocą instrukcji \textbf{popfq}. Ponieważ bufory liczb mają po 512 bajtów, a dodawanie jest wykonywane w rejestrach 64 bitowych, pętla dodawania wykona co najwyżej 64 iteracje.
\begin{verbatim}
add_numbers:
  clc           # Ustawienie flagi CF na 0
  pushfq        # Umieszczenie rejestru flag na stosie
  mov $0, %rsi
  
  add_loop:
    mov FIRST_NUM_BUFF(, %rsi, 8), %rax
    mov SECOND_NUM_BUFF(, %rsi, 8), %rbx
    popfq           # Pobranie rejestru flag ze stotsu
    adc %rbx, %rax  # Dodanie liczb z uwzględnieniem flagi przeniesienia
    pushfq          # Umieszczenie rejestru flag na stosie
    mov %rax, OUT_BUF(,%rsi, 8)
    inc %rsi
    cmp $64, %rsi
    jl add_loop
\end{verbatim}
\subsection{Konwersja na reprezentację szesnastkową}
Ponieważ wynik dodawania jest zapisany jako ciąg binarny, konwersja jest stosunkowa prosta - należy pogrupować ten ciąg w grupy po 4 bity, i wstawić znak odpowiadający danej cyfrze w grupie do bufora wyjściowego.
\begin{verbatim}
convert_loop:
  mov OUT_BUF(,%rdi, 1), %al     # Wstawienie kolejnego bajtu do %al
  mov %al, %bl						
  shr $4, %bl
  and $0b1111, %al               # W %al znajdą się pierwsze 4 bity  
  and $0b1111, %bl               # a w %bl ostatnie
  mov HEX_CHARS(,%rax,1), %cl    # Wstawienie znaku odpowiadającego cyfrze do %cl
  mov %cl, OUT_HEX_BUF(,%rsi,1)  # Wstawienie %cl do bufora wyjściowego
  inc %rsi
  mov HEX_CHARS(,%rbx,1), %cl
  mov %cl, OUT_HEX_BUF(,%rsi,1)
  inc %rsi
  inc %rdi
  cmp $512, %rdi
  jl convert_loop
\end{verbatim}
\subsection{Odwrócenie bufora wyjściowego}
W buforze wyjściowym znajduje się teraz odwrócony wynik dodawania, który jest zakończony ciągiem zer (bufory traktowane są jak liczby, nawet jeśli są w nich same zera). W celu zapisania prawidłowego wyniku do pliku, należy pozbyć się tych zer, i odwrócić kolejność cyfr.
\begin{verbatim}
reverse_out:
  mov $0, %rsi             
  mov $512, %rdi
  mov $0, %rax
  mov $0, %rbx
  mov $0, %rcx          # Jeśli %rcx jest równy 1, znaczy że obecny znak jest
                        # cyfrą liczby, nawet jeśli jest równy 0
  reverse_loop:
    cmp $0 ,%rdi
    jl end_reverse
    mov OUT_HEX_BUF(,%rdi,1), %al
    dec %rdi
    cmp $'0', %al        # Jeśli znak nie jest zerem, to jest cyfrą liczby
    jne place_char
    cmp $0, $rcx         # Jeśli znak jest zerem, i %rcx jest równe 1, to jest
    je reverse_loop      # cyfrą liczby
    
    place_char:
      mov $1, %rcx
      mov %al, OUT_BUF(,%rsi,1)
      inc %rsi
      jmp reverse_loop
      
  end_reverse:
    movb $0x0A, OUT_BUF(,%rsi,1) # Wstawienie znaku nowej lini na koniec bufora
    inc %rsi
    mov $0, %rdi
    mov %rsi, %rbx      # Ostateczna długość bufora wyjściowego

\end{verbatim}
\newpage
\subsection{Zapis wyniku do pliku}
Zapisywanie do pliku za pomocą wywołania systemowego wymaga trzech argumentów: deskryptora pliku, do którego chcemy zapisać, wskaźnika na początek bufora, oraz rozmiaru tego bufora.
Deskryptor pliku uzyskujemy tak samo jak w przypadku pliku wejściowego, za pomocą wywołania \textbf{open}. Długość bufora została obliczona w poprzednim kroku, i znajduje się w rejestrze \textbf{\%rbx}.
\begin{verbatim}
open_out:
  # Otworzenie pliku wyjściowego za pomocą wywołania systemowego open
  mov $SYSOPEN,  %rax
  mov $FILE_OUT, %rdi 
  mov $FWRITE,   %rsi 
  mov $0644,     %rdx  # Właściciel musi mieć prawo do zapisu
  syscall
  mov %rax, %r9        # Zapisanie deskryptora pliku wyjściowego
  
write_out:
  # Zapis do pliku
  movq $SYSWRITE,    %rax
  movq %r9,          %rdi
  movq $OUT_BUF,     %rsi
  movq %rbx,         %rdx
  
close_out:
  # Zamknięcie pliku wyjściowego
  mov $SYSCLOSE, %rax # Pierwszy parametr - numer wywołania
  mov %r9, %rdi       # Drugi parametr - ID otwartego pliku
  mov $0, %rsi
  mov $0, %rdx
  syscall
\end{verbatim}
\section{Wnioski}
Podczas operacji na plikach, trzeba szczególnie uważać na ich uprawnienia, ponieważ można przypadkowo utworzyć plik, którego obecny użytkownik nie będzie mógł odczytać.\\
Zamiana kolejności z big endian na little endian, w retrospekcji, mogła być zrealizowana za pomocą gotowych instrkucji służących do tego celu takich jak \textbf{bswp}.
\label{sec:wnioski}
\end{document}
