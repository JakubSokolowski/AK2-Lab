\documentclass[11pt]{report}

\usepackage[T1]{fontenc}
\usepackage[polish]{babel}
\usepackage{mathptmx}
\usepackage[utf8]{inputenc}
\usepackage{lmodern}
\usepackage{enumitem}
\usepackage{setspace}
\usepackage{advdate}
\usepackage[none]{hyphenat}
\selectlanguage{polish}
\usepackage[left=2.5cm,top=2.5cm,right=2.5cm,bottom=2.5cm,bindingoffset=0.5cm]{geometry}



\title{Sprawozdanie AK2 lab2}

\date{08-04-2018}
\author{Jakub Sokołowski}
\newcommand{\LabNum}{1}
\setcounter{chapter}{\LabNum}
\begin{document}


\noindent
  Jakub Sokołowski, 226080
  \hfill Wrocław, dn.\ {\AdvanceDate[-1]\today}\\
\hfill
\noindent
WT-N-07\hfill  prowadzący: Aleksandra Postawka\\
\vspace{1cm}
\begin{center}
  \begin{Large}
  	Laboratorium Architektury Komputerów\\
    \emph{(\LabNum) Proste konstruckje programowe z użyciem\\
    instrukcji asemblera}
  \end{Large}
\end{center}


\section{Treść ćwiczenia}
\label{sec:tresc-cwiczenia}
\textbf{Zakres i program ćwiczenia:}\\
Wykorzystanie prostych konstrukcji (pętle, badanie warunków) do operacji na ciągach znaków podawanych przez użytkownika. Ciąg znaków jest traktowany jako liczba w systemie \textbf{U8}.
\textbf{Zrealizowanie zadania:}
\begin{itemize}[leftmargin=*]
	\item Wczytanie ze \textbf{stdin} liczb w reprezentacji \textbf{ U8 (ASCII)}
	\item Sprawdzenie wczytanego ciągu znaków
	\item Wypisanie liczby na \textbf{stdout} w reprezentacji \textbf{U6}
\end{itemize}
\section{Przebieg ćwiczenia}
\label{sec:przebieg-cwiczenia}
\subsection{Wczytanie i weryfikacja znaków ASCII}
Pierwszym etapem programu było odczytanie znaków \textbf{ASCII} ze standardowego strumienia wejściowego (\textbf{stdin}) i zapisanie ich w pamięci jako liczby, na których było by możliwe wykonywanie późniejszych operacji matematycznych. Znaki odczytano za pomocą funkcji systemowej \textbf{read}.
\begin{verbatim}
# Wstawienie argumentów potrzebnych do funkcji read do odpowiednich rejestrów 
# IN_BUF został zadeklarowany w sekcji .bss jako .lcomm IN_BUF, 512

movq $SYSREAD,     %rax    # %rax - 0 - kod funkcji read
movq $STDIN,       %rdi    # %rdi - deskryptor pliku
movq $IN_BUF,      %rsi    # %rsi - początek bufora
movq $BUFFOR_SIZE, %rdx    # %rdx - rozmiar bufora
syscall

movq %rax, %r8             # Zapisanie liczby odczytanych bajtów do %r8
dec %r8                    # Znak '\n' nie jest brany pod uwagę
\end{verbatim}
\newpage
\noindent
Po wczytaniu do bufora należy zweryfikować, czy znaki należą do prawidłowych znaków systemu \textbf{U8} - mogą to być tylko liczby w zakresie 0-7. Konwersja znaku \textbf{ASCII} na liczbę jest realizowana przez odjęcie wartości znaku '0' (\textbf{0x30}) od danego znaku. Otrzymana po odjęciu wartość musi znajdować się w zakresie 0-7. W przypadku natrafienia na nieprawidłowy znak, na \textbf{stdout} wypisywany jest komunikat o błędzie.
\begin{verbatim}
  movq $0, %rdi               # Reset licznika używanego w pętli
  
validation_loop:
  cmp %r8, %rdi               # Pętla zakończona, gdy osiągnięto przedostatni znak
  je negativity_check
  mov $0, %rax
  mov IN_BUF(, %rdi, 1), %al  # Zapisanie pojedynczego znaku z bufora do %al
  sub $0x30, %al              # Odjęcie wartości znaku '0' od znaku w %al
                              # w celu otrzymania cyfry

  cmp $8, %al                 # Wartość w %al musi być pomiędzy 0-7
  jge write_err_msg           # Błąd, gdy jest większa niż 7
  cmp $0, %al                 # lub mniejsza niż 0
  jl write_err_msg
  # Znak zweryfikowany, zapisanie go do tablicy NUM_ARR i zwiększenie licznika
  mov %al, NUM_ARR(, %rdi, 4)
  inc %rdi
  jmp validation_loop
\end{verbatim}
\subsection{Obsługa liczb ujemnych}
Do obliczenia wartości danej liczby w systemie \textbf{U6} potrzebna będzie wartość bezwzględna tej liczby. W przypadku liczby ujemnej w systemie uzupełnieniowym, będzie to wymagało obliczenia jej dopełnienia. Najpierw trzeba sprawdzić, czy dana liczba jest ujemna czy dodatnia.
\begin{verbatim}
negativity_check:
  mov $0, %rdi                   # Indeks wskazujący na pierwszą cyfre w tablicy
  cmp $3, NUM_ARR(, %rdi, 4)     # Jeśli jest w zakresie 4-7 - liczba jest ujemna
  jge is_negative
  jmp is_positive
\end{verbatim}
Informacja o znaku liczby jest umieszczana w \textbf{\%r11}. Jeśli jest ujemna, trzeba obliczyć jej dopełnienie, a następnie uzupełnienie.
\begin{verbatim}
complement_loop:
  # Pętla po wszystkich cyfrach w tablicy 
  cmp %r8, %rdi              
  je increment_last          # Zwiększenie ostatniej cyfry o 1
  mov $7, %eax               # Wstawienie największej cyfry o danej podstawie do %eax
  sub NUM_ARR(,%rdi,4), %eax # Odjęcie obecnej cyfry od największej - dopełnienie cyfry
  mov %eax, NUM_ARR(,%rdi,4) # Wstawienie dopełnienia do tablucy
  inc %rdi					
  jmp complement_loop
\end{verbatim}
\newpage
\noindent
Po obliczeniu dopełnienia należy obliczyć uzupełnienie - dodać 1 do ostatniej cyfry liczby. Jeśli po dodaniu cyfra wynosi 7, zostaje ona ustawiona na 0, a zwiększana jest poprzednia cyfra.
\begin{verbatim}
increment_last:
  # Zmniejszenie indeksu, tak aby wskazywał na ostatnią cyfrę NUM_ARR - odwrotna pętla
  dec %rdi 
  increment_loop:
    mov NUM_ARR(,%rdi,4), %eax  # Wstawienie obecnej cyfry do %eax
    cmp $7, %eax                # Sprawdzenie, czy jest maksymalną
    jne increment_digit	        # Jeśli nie, zwiększ ją
    mov $0, %eax                # Maksymalna - ustawienie na 0
    mov %eax, NUM_ARR(, %rdi,4) # Wstawienie z powrotem do tablicy
    dec %rdi                    # Zmniejszenie licznka
    jmp increment_loop          # Zwiększenie poprzedniej liczby

	increment_digit:
      # Zwiększenie obecnej liczby o 1 i wstawienie jej do tablicy
      inc %eax
      mov %eax, NUM_ARR(, %rdi,4)
      mov $0, %rdi
      jmp convert_to_decimal
\end{verbatim}
\subsection{Konwersja na U10}
Wartość bezwzględna liczby jest teraz zapisana jako tablica cyfr tej liczby. Wartość dziesiętna, to suma jej cyfr pomnożonych przez podstawę systemu, podniesioną do wagi danej cyfry.
\begin{verbatim}
convert_to_decimal:
  # Odwrotna pętla po NUM_ARR
  mov %r8, %rdi   # Wstaw rozmiar NUM_ARR do %rdi
  dec %rdi
  mov $0, %r9     # Wynik konwersji w %r9
  mov $1, %rsi    # Pierwsza wartość podstawa ^ waga jest równa 1
  mov %rsi, %r10  # Waga cyfry
  mov $0, %rax    
  mov $U8_BASE, %rbx
  
  add NUM_ARR(,%rdi, 4), %r9	# Dodanie ostatniej cyfry do wyniku
  dec %rdi
\end{verbatim}
\newpage
\noindent
Po wstawieniu wszystkich potrzebnych wartości do rejestrów i obliczeniu wartości na ostatniej pozycji, trzeba obliczyć wartości na pozostałych pozycjach.
\begin{verbatim}
multiplication_loop:
  cmp $0, %rdi
  jl convert_to_base
  mov NUM_ARR(,%rdi,4), %eax 
  # Mnożenie przez podstawę aż wartość na pozycji zostnie obliczona
  compute_base_exponent:	 
  cmp $0, %r10               
  je finished_multiplication
  mul %rbx                   # Pomnożenie obecnej wartości przez podstawę
  dec %r10					  # Zmniejszenie licznikas
  jmp compute_base_exponent
  
  finished_multiplication:  
    # Wartość cyfra * podstawa ^ waga znajdue się w %rax
    # Dodaj wartość na pozycji do wyniku końcowego    
    add %rax, %r9
    inc %rsi                 # Następna pozycja ma wagę większą o 1
    mov %rsi, %r10           # Umieść tą wartość w liczniku pętli
    dec %rdi                 # Zmniejsz indeks NUM_ARR
    jmp multiplication_loop
\end{verbatim}
\subsection{Konwersja na U6}
Wartość w systemie U6 obliczana za pomocą standardowego algorytmu i zapisywana do tablicy REV\_NUM\_ARR, którą następne należy odwrócić, by uzyskać prawidłowy wynik.
\begin{verbatim}
convert_to_base:
  # r8  - Rozmiar NUM_ARR
  # r9  - Wartość dziesiętna liczby
  # r10 - Podstawa U6
  # rdi - Pozycja w REV_ARR
  # r11 - Znak
  # rax - Wynik dzielenia
  # rdx - Reszta z dzielenia
  mov $0, %rdi       # Obecny indeks w REV_ARR to 0
  mov $U6_BASE, %r10 
  mov %r9, %rax      # Wstawienie liczby do %rax w celu wykonania dzielenia
\end{verbatim}
\newpage
\noindent
\begin{verbatim}
division_loop:
  mov $0, %edx                   # Wyzerowanie %edx przed dzieleniem
  div %r10                       # Dzielenie liczby przez podstawę
                                 # Wartość na pozycji to reszta z dzielenia - %rdx
  mov %edx, REV_NUM_ARR(,%rdi,4) # Wstawienie cyfry do REV_NUM_ARR
  inc %rdi
  cmp %rax, %r10                 # Sprawdzenie warunku zatrzymania wynik < podstawa
  jle division_loop              # Wynik > podstawa, kontynuacja pętli
  
  # Osiągnięty warunek zatrzymania - do REV_NUM_ARR wstawiony jest wynik z dzielenia
  mov %eax, REV_NUM_ARR(,%rdi,4)
  mov %rdi, %r8                  # Zapisanie rozmiaru tablicy
  
# Wpisanie wartości z REV_NUM_ARR w odwrotnej kolejności do NUM_ARR
reverse_array:
  mov $0 ,%eax
  mov $0, %rsi
  
  reverse_loop:
    cmp $0, %rdi
    jl complement_result
    mov REV_NUM_ARR(,%rdi,4), %eax
    mov %eax, NUM_ARR(,%rsi,4)
    inc %rsi
    dec %rdi
    jmp reverse_loop
\end{verbatim}
W przypadku liczb dodatnich pozostaje jedynie wypisanie wyniku konwersji na ekran, natomiast dla liczb ujemnych trzeba jeszcze raz obliczyć uzupełnienie.
\newpage
\noindent
\subsection{Wypisanie wyniku na wyjście}
Istnieją 3 warianty zakończenia programu - wypisania komunikatu o błędzie, wypisanie liczby dodatniej i wypisanie liczby ujemnej. W przypadku liczby ujemnej przed wartością jest wypisywane rozszerzenie - dla systemu \textbf{U6} jest to '(5)'. Pisanie do \textbf{stdout} jest realizowane za pomocą funkcji sytemowej \textbf{write}. Przed wypisaniem należy przekonwertować cyfry na \textbf{ASCII}, co uzyskiwane jest przez dodanie wartości znaku '0' do każdej cyfry.
\begin{verbatim}
int_to_ascii:
  mov $0, %rdi
  ascii_loop:
    cmp %r8, %rdi               
    jg write_outpt
    mov $0, %rax
    mov NUM_ARR(, %rdi, 4), %rax  
    add $0x30, %rax               
    mov %al, OUT_BUF(, %rdi, 1)
    inc %rdi
    jmp ascii_loop
\end{verbatim}
Jako że funkcja write pojawia się za każdym razem, zostanie opisany tylko przypadek l. ujemnej.
\begin{verbatim}
# Wypisanie zawartości OUT_BUF na stdout
write_outpt:
  mov %rdi, %r10
  cmp $0, %r11
  je after_prefix
  movq $SYSWRITE,     %rax   # %rax - 1 - kod funkcji write
  movq $STDOUT,       %rdi   # %rdi - deskryptor pliku
  movq $prefix,       %rsi   # %rsi - początek bufora
  movq $prefix_size,  %rdx   # %rdx - rozmiar bufora
  syscall

after_prefix:
  mov %r10, %rdi
  movb $NEWLINE, OUT_BUF(, %rdi, 1)
  movq $SYSWRITE,    
  movq $STDOUT,       
  movq $OUT_BUF,      
  movq $BUFFOR_SIZE,  
  syscall

  movq $SYSWRITE,   
  movq $STDOUT,      
  movq $NEWLINE,     
  movq $1,  %rdx
  syscall
  jmp end
\end{verbatim}
\subsection{Uruchamianie programu}
Program został skompilowany i skonsolidowany za pomocą poniższego makefila:
\begin{verbatim}
lab1 : lab1.o
  ld  -o lab1 lab1.o
lab1.o : lab1.s
  as -gstabs -o lab1.o lab1.s
\end{verbatim}
W celu umożliwienia wygodniejszej pracy z debuggerem \textbf{gdb} została użyta flaga \textbf{-gstabs} która generuje informacje do debuggowania (talicę powiązań symboli).
\section{Wnioski}
Obliczanie wartości dziesiętnej wprowadzanej liczby można by było zrealizować wraz z weryfikowaniem cyfr. Użycie funkcji do obliczania uzupełnienia liczby znacznie zmniejszyło by długość kodu programu i jednocześnie zwiększyło by czytelność. Warto też zauważyć, że na końcu tekstu wczytanego ze \textbf{stdin} znajduje się znak  nowej linii, co należy wziąć pod uwagę podczas indeksowania tego tekstu.
\label{sec:wnioski}
\end{document}
